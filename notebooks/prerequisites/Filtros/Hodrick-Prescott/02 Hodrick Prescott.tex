\documentclass[11pt]{article}

    \usepackage[breakable]{tcolorbox}
    \usepackage{parskip} % Stop auto-indenting (to mimic markdown behaviour)
    

    % Basic figure setup, for now with no caption control since it's done
    % automatically by Pandoc (which extracts ![](path) syntax from Markdown).
    \usepackage{graphicx}
    % Keep aspect ratio if custom image width or height is specified
    \setkeys{Gin}{keepaspectratio}
    % Maintain compatibility with old templates. Remove in nbconvert 6.0
    \let\Oldincludegraphics\includegraphics
    % Ensure that by default, figures have no caption (until we provide a
    % proper Figure object with a Caption API and a way to capture that
    % in the conversion process - todo).
    \usepackage{caption}
    \DeclareCaptionFormat{nocaption}{}
    \captionsetup{format=nocaption,aboveskip=0pt,belowskip=0pt}

    \usepackage{float}
    \floatplacement{figure}{H} % forces figures to be placed at the correct location
    \usepackage{xcolor} % Allow colors to be defined
    \usepackage{enumerate} % Needed for markdown enumerations to work
    \usepackage{geometry} % Used to adjust the document margins
    \usepackage{amsmath} % Equations
    \usepackage{amssymb} % Equations
    \usepackage{textcomp} % defines textquotesingle
    % Hack from http://tex.stackexchange.com/a/47451/13684:
    \AtBeginDocument{%
        \def\PYZsq{\textquotesingle}% Upright quotes in Pygmentized code
    }
    \usepackage{upquote} % Upright quotes for verbatim code
    \usepackage{eurosym} % defines \euro

    \usepackage{iftex}
    \ifPDFTeX
        \usepackage[T1]{fontenc}
        \IfFileExists{alphabeta.sty}{
              \usepackage{alphabeta}
          }{
              \usepackage[mathletters]{ucs}
              \usepackage[utf8x]{inputenc}
          }
    \else
        \usepackage{fontspec}
        \usepackage{unicode-math}
    \fi

    \usepackage{fancyvrb} % verbatim replacement that allows latex
    \usepackage{grffile} % extends the file name processing of package graphics
                         % to support a larger range
    \makeatletter % fix for old versions of grffile with XeLaTeX
    \@ifpackagelater{grffile}{2019/11/01}
    {
      % Do nothing on new versions
    }
    {
      \def\Gread@@xetex#1{%
        \IfFileExists{"\Gin@base".bb}%
        {\Gread@eps{\Gin@base.bb}}%
        {\Gread@@xetex@aux#1}%
      }
    }
    \makeatother
    \usepackage[Export]{adjustbox} % Used to constrain images to a maximum size
    \adjustboxset{max size={0.9\linewidth}{0.9\paperheight}}

    % The hyperref package gives us a pdf with properly built
    % internal navigation ('pdf bookmarks' for the table of contents,
    % internal cross-reference links, web links for URLs, etc.)
    \usepackage{hyperref}
    % The default LaTeX title has an obnoxious amount of whitespace. By default,
    % titling removes some of it. It also provides customization options.
    \usepackage{titling}
    \usepackage{longtable} % longtable support required by pandoc >1.10
    \usepackage{booktabs}  % table support for pandoc > 1.12.2
    \usepackage{array}     % table support for pandoc >= 2.11.3
    \usepackage{calc}      % table minipage width calculation for pandoc >= 2.11.1
    \usepackage[inline]{enumitem} % IRkernel/repr support (it uses the enumerate* environment)
    \usepackage[normalem]{ulem} % ulem is needed to support strikethroughs (\sout)
                                % normalem makes italics be italics, not underlines
    \usepackage{soul}      % strikethrough (\st) support for pandoc >= 3.0.0
    \usepackage{mathrsfs}
    

    
    % Colors for the hyperref package
    \definecolor{urlcolor}{rgb}{0,.145,.698}
    \definecolor{linkcolor}{rgb}{.71,0.21,0.01}
    \definecolor{citecolor}{rgb}{.12,.54,.11}

    % ANSI colors
    \definecolor{ansi-black}{HTML}{3E424D}
    \definecolor{ansi-black-intense}{HTML}{282C36}
    \definecolor{ansi-red}{HTML}{E75C58}
    \definecolor{ansi-red-intense}{HTML}{B22B31}
    \definecolor{ansi-green}{HTML}{00A250}
    \definecolor{ansi-green-intense}{HTML}{007427}
    \definecolor{ansi-yellow}{HTML}{DDB62B}
    \definecolor{ansi-yellow-intense}{HTML}{B27D12}
    \definecolor{ansi-blue}{HTML}{208FFB}
    \definecolor{ansi-blue-intense}{HTML}{0065CA}
    \definecolor{ansi-magenta}{HTML}{D160C4}
    \definecolor{ansi-magenta-intense}{HTML}{A03196}
    \definecolor{ansi-cyan}{HTML}{60C6C8}
    \definecolor{ansi-cyan-intense}{HTML}{258F8F}
    \definecolor{ansi-white}{HTML}{C5C1B4}
    \definecolor{ansi-white-intense}{HTML}{A1A6B2}
    \definecolor{ansi-default-inverse-fg}{HTML}{FFFFFF}
    \definecolor{ansi-default-inverse-bg}{HTML}{000000}

    % common color for the border for error outputs.
    \definecolor{outerrorbackground}{HTML}{FFDFDF}

    % commands and environments needed by pandoc snippets
    % extracted from the output of `pandoc -s`
    \providecommand{\tightlist}{%
      \setlength{\itemsep}{0pt}\setlength{\parskip}{0pt}}
    \DefineVerbatimEnvironment{Highlighting}{Verbatim}{commandchars=\\\{\}}
    % Add ',fontsize=\small' for more characters per line
    \newenvironment{Shaded}{}{}
    \newcommand{\KeywordTok}[1]{\textcolor[rgb]{0.00,0.44,0.13}{\textbf{{#1}}}}
    \newcommand{\DataTypeTok}[1]{\textcolor[rgb]{0.56,0.13,0.00}{{#1}}}
    \newcommand{\DecValTok}[1]{\textcolor[rgb]{0.25,0.63,0.44}{{#1}}}
    \newcommand{\BaseNTok}[1]{\textcolor[rgb]{0.25,0.63,0.44}{{#1}}}
    \newcommand{\FloatTok}[1]{\textcolor[rgb]{0.25,0.63,0.44}{{#1}}}
    \newcommand{\CharTok}[1]{\textcolor[rgb]{0.25,0.44,0.63}{{#1}}}
    \newcommand{\StringTok}[1]{\textcolor[rgb]{0.25,0.44,0.63}{{#1}}}
    \newcommand{\CommentTok}[1]{\textcolor[rgb]{0.38,0.63,0.69}{\textit{{#1}}}}
    \newcommand{\OtherTok}[1]{\textcolor[rgb]{0.00,0.44,0.13}{{#1}}}
    \newcommand{\AlertTok}[1]{\textcolor[rgb]{1.00,0.00,0.00}{\textbf{{#1}}}}
    \newcommand{\FunctionTok}[1]{\textcolor[rgb]{0.02,0.16,0.49}{{#1}}}
    \newcommand{\RegionMarkerTok}[1]{{#1}}
    \newcommand{\ErrorTok}[1]{\textcolor[rgb]{1.00,0.00,0.00}{\textbf{{#1}}}}
    \newcommand{\NormalTok}[1]{{#1}}

    % Additional commands for more recent versions of Pandoc
    \newcommand{\ConstantTok}[1]{\textcolor[rgb]{0.53,0.00,0.00}{{#1}}}
    \newcommand{\SpecialCharTok}[1]{\textcolor[rgb]{0.25,0.44,0.63}{{#1}}}
    \newcommand{\VerbatimStringTok}[1]{\textcolor[rgb]{0.25,0.44,0.63}{{#1}}}
    \newcommand{\SpecialStringTok}[1]{\textcolor[rgb]{0.73,0.40,0.53}{{#1}}}
    \newcommand{\ImportTok}[1]{{#1}}
    \newcommand{\DocumentationTok}[1]{\textcolor[rgb]{0.73,0.13,0.13}{\textit{{#1}}}}
    \newcommand{\AnnotationTok}[1]{\textcolor[rgb]{0.38,0.63,0.69}{\textbf{\textit{{#1}}}}}
    \newcommand{\CommentVarTok}[1]{\textcolor[rgb]{0.38,0.63,0.69}{\textbf{\textit{{#1}}}}}
    \newcommand{\VariableTok}[1]{\textcolor[rgb]{0.10,0.09,0.49}{{#1}}}
    \newcommand{\ControlFlowTok}[1]{\textcolor[rgb]{0.00,0.44,0.13}{\textbf{{#1}}}}
    \newcommand{\OperatorTok}[1]{\textcolor[rgb]{0.40,0.40,0.40}{{#1}}}
    \newcommand{\BuiltInTok}[1]{{#1}}
    \newcommand{\ExtensionTok}[1]{{#1}}
    \newcommand{\PreprocessorTok}[1]{\textcolor[rgb]{0.74,0.48,0.00}{{#1}}}
    \newcommand{\AttributeTok}[1]{\textcolor[rgb]{0.49,0.56,0.16}{{#1}}}
    \newcommand{\InformationTok}[1]{\textcolor[rgb]{0.38,0.63,0.69}{\textbf{\textit{{#1}}}}}
    \newcommand{\WarningTok}[1]{\textcolor[rgb]{0.38,0.63,0.69}{\textbf{\textit{{#1}}}}}
    \makeatletter
    \newsavebox\pandoc@box
    \newcommand*\pandocbounded[1]{%
      \sbox\pandoc@box{#1}%
      % scaling factors for width and height
      \Gscale@div\@tempa\textheight{\dimexpr\ht\pandoc@box+\dp\pandoc@box\relax}%
      \Gscale@div\@tempb\linewidth{\wd\pandoc@box}%
      % select the smaller of both
      \ifdim\@tempb\p@<\@tempa\p@
        \let\@tempa\@tempb
      \fi
      % scaling accordingly (\@tempa < 1)
      \ifdim\@tempa\p@<\p@
        \scalebox{\@tempa}{\usebox\pandoc@box}%
      % scaling not needed, use as it is
      \else
        \usebox{\pandoc@box}%
      \fi
    }
    \makeatother

    % Define a nice break command that doesn't care if a line doesn't already
    % exist.
    \def\br{\hspace*{\fill} \\* }
    % Math Jax compatibility definitions
    \def\gt{>}
    \def\lt{<}
    \let\Oldtex\TeX
    \let\Oldlatex\LaTeX
    \renewcommand{\TeX}{\textrm{\Oldtex}}
    \renewcommand{\LaTeX}{\textrm{\Oldlatex}}
    % Document parameters
    % Document title
    \title{02 Hodrick Prescott}
    
    
    
    
    
    
    
% Pygments definitions
\makeatletter
\def\PY@reset{\let\PY@it=\relax \let\PY@bf=\relax%
    \let\PY@ul=\relax \let\PY@tc=\relax%
    \let\PY@bc=\relax \let\PY@ff=\relax}
\def\PY@tok#1{\csname PY@tok@#1\endcsname}
\def\PY@toks#1+{\ifx\relax#1\empty\else%
    \PY@tok{#1}\expandafter\PY@toks\fi}
\def\PY@do#1{\PY@bc{\PY@tc{\PY@ul{%
    \PY@it{\PY@bf{\PY@ff{#1}}}}}}}
\def\PY#1#2{\PY@reset\PY@toks#1+\relax+\PY@do{#2}}

\@namedef{PY@tok@w}{\def\PY@tc##1{\textcolor[rgb]{0.73,0.73,0.73}{##1}}}
\@namedef{PY@tok@c}{\let\PY@it=\textit\def\PY@tc##1{\textcolor[rgb]{0.24,0.48,0.48}{##1}}}
\@namedef{PY@tok@cp}{\def\PY@tc##1{\textcolor[rgb]{0.61,0.40,0.00}{##1}}}
\@namedef{PY@tok@k}{\let\PY@bf=\textbf\def\PY@tc##1{\textcolor[rgb]{0.00,0.50,0.00}{##1}}}
\@namedef{PY@tok@kp}{\def\PY@tc##1{\textcolor[rgb]{0.00,0.50,0.00}{##1}}}
\@namedef{PY@tok@kt}{\def\PY@tc##1{\textcolor[rgb]{0.69,0.00,0.25}{##1}}}
\@namedef{PY@tok@o}{\def\PY@tc##1{\textcolor[rgb]{0.40,0.40,0.40}{##1}}}
\@namedef{PY@tok@ow}{\let\PY@bf=\textbf\def\PY@tc##1{\textcolor[rgb]{0.67,0.13,1.00}{##1}}}
\@namedef{PY@tok@nb}{\def\PY@tc##1{\textcolor[rgb]{0.00,0.50,0.00}{##1}}}
\@namedef{PY@tok@nf}{\def\PY@tc##1{\textcolor[rgb]{0.00,0.00,1.00}{##1}}}
\@namedef{PY@tok@nc}{\let\PY@bf=\textbf\def\PY@tc##1{\textcolor[rgb]{0.00,0.00,1.00}{##1}}}
\@namedef{PY@tok@nn}{\let\PY@bf=\textbf\def\PY@tc##1{\textcolor[rgb]{0.00,0.00,1.00}{##1}}}
\@namedef{PY@tok@ne}{\let\PY@bf=\textbf\def\PY@tc##1{\textcolor[rgb]{0.80,0.25,0.22}{##1}}}
\@namedef{PY@tok@nv}{\def\PY@tc##1{\textcolor[rgb]{0.10,0.09,0.49}{##1}}}
\@namedef{PY@tok@no}{\def\PY@tc##1{\textcolor[rgb]{0.53,0.00,0.00}{##1}}}
\@namedef{PY@tok@nl}{\def\PY@tc##1{\textcolor[rgb]{0.46,0.46,0.00}{##1}}}
\@namedef{PY@tok@ni}{\let\PY@bf=\textbf\def\PY@tc##1{\textcolor[rgb]{0.44,0.44,0.44}{##1}}}
\@namedef{PY@tok@na}{\def\PY@tc##1{\textcolor[rgb]{0.41,0.47,0.13}{##1}}}
\@namedef{PY@tok@nt}{\let\PY@bf=\textbf\def\PY@tc##1{\textcolor[rgb]{0.00,0.50,0.00}{##1}}}
\@namedef{PY@tok@nd}{\def\PY@tc##1{\textcolor[rgb]{0.67,0.13,1.00}{##1}}}
\@namedef{PY@tok@s}{\def\PY@tc##1{\textcolor[rgb]{0.73,0.13,0.13}{##1}}}
\@namedef{PY@tok@sd}{\let\PY@it=\textit\def\PY@tc##1{\textcolor[rgb]{0.73,0.13,0.13}{##1}}}
\@namedef{PY@tok@si}{\let\PY@bf=\textbf\def\PY@tc##1{\textcolor[rgb]{0.64,0.35,0.47}{##1}}}
\@namedef{PY@tok@se}{\let\PY@bf=\textbf\def\PY@tc##1{\textcolor[rgb]{0.67,0.36,0.12}{##1}}}
\@namedef{PY@tok@sr}{\def\PY@tc##1{\textcolor[rgb]{0.64,0.35,0.47}{##1}}}
\@namedef{PY@tok@ss}{\def\PY@tc##1{\textcolor[rgb]{0.10,0.09,0.49}{##1}}}
\@namedef{PY@tok@sx}{\def\PY@tc##1{\textcolor[rgb]{0.00,0.50,0.00}{##1}}}
\@namedef{PY@tok@m}{\def\PY@tc##1{\textcolor[rgb]{0.40,0.40,0.40}{##1}}}
\@namedef{PY@tok@gh}{\let\PY@bf=\textbf\def\PY@tc##1{\textcolor[rgb]{0.00,0.00,0.50}{##1}}}
\@namedef{PY@tok@gu}{\let\PY@bf=\textbf\def\PY@tc##1{\textcolor[rgb]{0.50,0.00,0.50}{##1}}}
\@namedef{PY@tok@gd}{\def\PY@tc##1{\textcolor[rgb]{0.63,0.00,0.00}{##1}}}
\@namedef{PY@tok@gi}{\def\PY@tc##1{\textcolor[rgb]{0.00,0.52,0.00}{##1}}}
\@namedef{PY@tok@gr}{\def\PY@tc##1{\textcolor[rgb]{0.89,0.00,0.00}{##1}}}
\@namedef{PY@tok@ge}{\let\PY@it=\textit}
\@namedef{PY@tok@gs}{\let\PY@bf=\textbf}
\@namedef{PY@tok@ges}{\let\PY@bf=\textbf\let\PY@it=\textit}
\@namedef{PY@tok@gp}{\let\PY@bf=\textbf\def\PY@tc##1{\textcolor[rgb]{0.00,0.00,0.50}{##1}}}
\@namedef{PY@tok@go}{\def\PY@tc##1{\textcolor[rgb]{0.44,0.44,0.44}{##1}}}
\@namedef{PY@tok@gt}{\def\PY@tc##1{\textcolor[rgb]{0.00,0.27,0.87}{##1}}}
\@namedef{PY@tok@err}{\def\PY@bc##1{{\setlength{\fboxsep}{\string -\fboxrule}\fcolorbox[rgb]{1.00,0.00,0.00}{1,1,1}{\strut ##1}}}}
\@namedef{PY@tok@kc}{\let\PY@bf=\textbf\def\PY@tc##1{\textcolor[rgb]{0.00,0.50,0.00}{##1}}}
\@namedef{PY@tok@kd}{\let\PY@bf=\textbf\def\PY@tc##1{\textcolor[rgb]{0.00,0.50,0.00}{##1}}}
\@namedef{PY@tok@kn}{\let\PY@bf=\textbf\def\PY@tc##1{\textcolor[rgb]{0.00,0.50,0.00}{##1}}}
\@namedef{PY@tok@kr}{\let\PY@bf=\textbf\def\PY@tc##1{\textcolor[rgb]{0.00,0.50,0.00}{##1}}}
\@namedef{PY@tok@bp}{\def\PY@tc##1{\textcolor[rgb]{0.00,0.50,0.00}{##1}}}
\@namedef{PY@tok@fm}{\def\PY@tc##1{\textcolor[rgb]{0.00,0.00,1.00}{##1}}}
\@namedef{PY@tok@vc}{\def\PY@tc##1{\textcolor[rgb]{0.10,0.09,0.49}{##1}}}
\@namedef{PY@tok@vg}{\def\PY@tc##1{\textcolor[rgb]{0.10,0.09,0.49}{##1}}}
\@namedef{PY@tok@vi}{\def\PY@tc##1{\textcolor[rgb]{0.10,0.09,0.49}{##1}}}
\@namedef{PY@tok@vm}{\def\PY@tc##1{\textcolor[rgb]{0.10,0.09,0.49}{##1}}}
\@namedef{PY@tok@sa}{\def\PY@tc##1{\textcolor[rgb]{0.73,0.13,0.13}{##1}}}
\@namedef{PY@tok@sb}{\def\PY@tc##1{\textcolor[rgb]{0.73,0.13,0.13}{##1}}}
\@namedef{PY@tok@sc}{\def\PY@tc##1{\textcolor[rgb]{0.73,0.13,0.13}{##1}}}
\@namedef{PY@tok@dl}{\def\PY@tc##1{\textcolor[rgb]{0.73,0.13,0.13}{##1}}}
\@namedef{PY@tok@s2}{\def\PY@tc##1{\textcolor[rgb]{0.73,0.13,0.13}{##1}}}
\@namedef{PY@tok@sh}{\def\PY@tc##1{\textcolor[rgb]{0.73,0.13,0.13}{##1}}}
\@namedef{PY@tok@s1}{\def\PY@tc##1{\textcolor[rgb]{0.73,0.13,0.13}{##1}}}
\@namedef{PY@tok@mb}{\def\PY@tc##1{\textcolor[rgb]{0.40,0.40,0.40}{##1}}}
\@namedef{PY@tok@mf}{\def\PY@tc##1{\textcolor[rgb]{0.40,0.40,0.40}{##1}}}
\@namedef{PY@tok@mh}{\def\PY@tc##1{\textcolor[rgb]{0.40,0.40,0.40}{##1}}}
\@namedef{PY@tok@mi}{\def\PY@tc##1{\textcolor[rgb]{0.40,0.40,0.40}{##1}}}
\@namedef{PY@tok@il}{\def\PY@tc##1{\textcolor[rgb]{0.40,0.40,0.40}{##1}}}
\@namedef{PY@tok@mo}{\def\PY@tc##1{\textcolor[rgb]{0.40,0.40,0.40}{##1}}}
\@namedef{PY@tok@ch}{\let\PY@it=\textit\def\PY@tc##1{\textcolor[rgb]{0.24,0.48,0.48}{##1}}}
\@namedef{PY@tok@cm}{\let\PY@it=\textit\def\PY@tc##1{\textcolor[rgb]{0.24,0.48,0.48}{##1}}}
\@namedef{PY@tok@cpf}{\let\PY@it=\textit\def\PY@tc##1{\textcolor[rgb]{0.24,0.48,0.48}{##1}}}
\@namedef{PY@tok@c1}{\let\PY@it=\textit\def\PY@tc##1{\textcolor[rgb]{0.24,0.48,0.48}{##1}}}
\@namedef{PY@tok@cs}{\let\PY@it=\textit\def\PY@tc##1{\textcolor[rgb]{0.24,0.48,0.48}{##1}}}

\def\PYZbs{\char`\\}
\def\PYZus{\char`\_}
\def\PYZob{\char`\{}
\def\PYZcb{\char`\}}
\def\PYZca{\char`\^}
\def\PYZam{\char`\&}
\def\PYZlt{\char`\<}
\def\PYZgt{\char`\>}
\def\PYZsh{\char`\#}
\def\PYZpc{\char`\%}
\def\PYZdl{\char`\$}
\def\PYZhy{\char`\-}
\def\PYZsq{\char`\'}
\def\PYZdq{\char`\"}
\def\PYZti{\char`\~}
% for compatibility with earlier versions
\def\PYZat{@}
\def\PYZlb{[}
\def\PYZrb{]}
\makeatother


    % For linebreaks inside Verbatim environment from package fancyvrb.
    \makeatletter
        \newbox\Wrappedcontinuationbox
        \newbox\Wrappedvisiblespacebox
        \newcommand*\Wrappedvisiblespace {\textcolor{red}{\textvisiblespace}}
        \newcommand*\Wrappedcontinuationsymbol {\textcolor{red}{\llap{\tiny$\m@th\hookrightarrow$}}}
        \newcommand*\Wrappedcontinuationindent {3ex }
        \newcommand*\Wrappedafterbreak {\kern\Wrappedcontinuationindent\copy\Wrappedcontinuationbox}
        % Take advantage of the already applied Pygments mark-up to insert
        % potential linebreaks for TeX processing.
        %        {, <, #, %, $, ' and ": go to next line.
        %        _, }, ^, &, >, - and ~: stay at end of broken line.
        % Use of \textquotesingle for straight quote.
        \newcommand*\Wrappedbreaksatspecials {%
            \def\PYGZus{\discretionary{\char`\_}{\Wrappedafterbreak}{\char`\_}}%
            \def\PYGZob{\discretionary{}{\Wrappedafterbreak\char`\{}{\char`\{}}%
            \def\PYGZcb{\discretionary{\char`\}}{\Wrappedafterbreak}{\char`\}}}%
            \def\PYGZca{\discretionary{\char`\^}{\Wrappedafterbreak}{\char`\^}}%
            \def\PYGZam{\discretionary{\char`\&}{\Wrappedafterbreak}{\char`\&}}%
            \def\PYGZlt{\discretionary{}{\Wrappedafterbreak\char`\<}{\char`\<}}%
            \def\PYGZgt{\discretionary{\char`\>}{\Wrappedafterbreak}{\char`\>}}%
            \def\PYGZsh{\discretionary{}{\Wrappedafterbreak\char`\#}{\char`\#}}%
            \def\PYGZpc{\discretionary{}{\Wrappedafterbreak\char`\%}{\char`\%}}%
            \def\PYGZdl{\discretionary{}{\Wrappedafterbreak\char`\$}{\char`\$}}%
            \def\PYGZhy{\discretionary{\char`\-}{\Wrappedafterbreak}{\char`\-}}%
            \def\PYGZsq{\discretionary{}{\Wrappedafterbreak\textquotesingle}{\textquotesingle}}%
            \def\PYGZdq{\discretionary{}{\Wrappedafterbreak\char`\"}{\char`\"}}%
            \def\PYGZti{\discretionary{\char`\~}{\Wrappedafterbreak}{\char`\~}}%
        }
        % Some characters . , ; ? ! / are not pygmentized.
        % This macro makes them "active" and they will insert potential linebreaks
        \newcommand*\Wrappedbreaksatpunct {%
            \lccode`\~`\.\lowercase{\def~}{\discretionary{\hbox{\char`\.}}{\Wrappedafterbreak}{\hbox{\char`\.}}}%
            \lccode`\~`\,\lowercase{\def~}{\discretionary{\hbox{\char`\,}}{\Wrappedafterbreak}{\hbox{\char`\,}}}%
            \lccode`\~`\;\lowercase{\def~}{\discretionary{\hbox{\char`\;}}{\Wrappedafterbreak}{\hbox{\char`\;}}}%
            \lccode`\~`\:\lowercase{\def~}{\discretionary{\hbox{\char`\:}}{\Wrappedafterbreak}{\hbox{\char`\:}}}%
            \lccode`\~`\?\lowercase{\def~}{\discretionary{\hbox{\char`\?}}{\Wrappedafterbreak}{\hbox{\char`\?}}}%
            \lccode`\~`\!\lowercase{\def~}{\discretionary{\hbox{\char`\!}}{\Wrappedafterbreak}{\hbox{\char`\!}}}%
            \lccode`\~`\/\lowercase{\def~}{\discretionary{\hbox{\char`\/}}{\Wrappedafterbreak}{\hbox{\char`\/}}}%
            \catcode`\.\active
            \catcode`\,\active
            \catcode`\;\active
            \catcode`\:\active
            \catcode`\?\active
            \catcode`\!\active
            \catcode`\/\active
            \lccode`\~`\~
        }
    \makeatother

    \let\OriginalVerbatim=\Verbatim
    \makeatletter
    \renewcommand{\Verbatim}[1][1]{%
        %\parskip\z@skip
        \sbox\Wrappedcontinuationbox {\Wrappedcontinuationsymbol}%
        \sbox\Wrappedvisiblespacebox {\FV@SetupFont\Wrappedvisiblespace}%
        \def\FancyVerbFormatLine ##1{\hsize\linewidth
            \vtop{\raggedright\hyphenpenalty\z@\exhyphenpenalty\z@
                \doublehyphendemerits\z@\finalhyphendemerits\z@
                \strut ##1\strut}%
        }%
        % If the linebreak is at a space, the latter will be displayed as visible
        % space at end of first line, and a continuation symbol starts next line.
        % Stretch/shrink are however usually zero for typewriter font.
        \def\FV@Space {%
            \nobreak\hskip\z@ plus\fontdimen3\font minus\fontdimen4\font
            \discretionary{\copy\Wrappedvisiblespacebox}{\Wrappedafterbreak}
            {\kern\fontdimen2\font}%
        }%

        % Allow breaks at special characters using \PYG... macros.
        \Wrappedbreaksatspecials
        % Breaks at punctuation characters . , ; ? ! and / need catcode=\active
        \OriginalVerbatim[#1,codes*=\Wrappedbreaksatpunct]%
    }
    \makeatother

    % Exact colors from NB
    \definecolor{incolor}{HTML}{303F9F}
    \definecolor{outcolor}{HTML}{D84315}
    \definecolor{cellborder}{HTML}{CFCFCF}
    \definecolor{cellbackground}{HTML}{F7F7F7}

    % prompt
    \makeatletter
    \newcommand{\boxspacing}{\kern\kvtcb@left@rule\kern\kvtcb@boxsep}
    \makeatother
    \newcommand{\prompt}[4]{
        {\ttfamily\llap{{\color{#2}[#3]:\hspace{3pt}#4}}\vspace{-\baselineskip}}
    }
    

    
    % Prevent overflowing lines due to hard-to-break entities
    \sloppy
    % Setup hyperref package
    \hypersetup{
      breaklinks=true,  % so long urls are correctly broken across lines
      colorlinks=true,
      urlcolor=urlcolor,
      linkcolor=linkcolor,
      citecolor=citecolor,
      }
    % Slightly bigger margins than the latex defaults
    
    \geometry{verbose,tmargin=1in,bmargin=1in,lmargin=1in,rmargin=1in}
    
    

\begin{document}
    
    \maketitle
    
    

    
    \texttt{\{include\}\ ../math-definitions.md}

    \begin{tcolorbox}[breakable, size=fbox, boxrule=1pt, pad at break*=1mm,colback=cellbackground, colframe=cellborder]
\prompt{In}{incolor}{1}{\boxspacing}
\begin{Verbatim}[commandchars=\\\{\}]
\PY{k+kn}{import}\PY{+w}{ }\PY{n+nn}{numpy}\PY{+w}{ }\PY{k}{as}\PY{+w}{ }\PY{n+nn}{np}
\PY{k+kn}{import}\PY{+w}{ }\PY{n+nn}{pandas}\PY{+w}{ }\PY{k}{as}\PY{+w}{ }\PY{n+nn}{pd}
\PY{n}{pd}\PY{o}{.}\PY{n}{options}\PY{o}{.}\PY{n}{plotting}\PY{o}{.}\PY{n}{backend} \PY{o}{=} \PY{l+s+s2}{\PYZdq{}}\PY{l+s+s2}{plotly}\PY{l+s+s2}{\PYZdq{}}
\PY{k+kn}{import}\PY{+w}{ }\PY{n+nn}{pandas\PYZus{}datareader}\PY{+w}{ }\PY{k}{as}\PY{+w}{ }\PY{n+nn}{pdr}
\end{Verbatim}
\end{tcolorbox}

    \hypertarget{el-filtro-de-hodrick-y-prescott}{%
\section{El filtro de Hodrick y
Prescott}\label{el-filtro-de-hodrick-y-prescott}}

\textbf{Otro método para remover una tendencia}

\hypertarget{desagregaciuxf3n-de-una-serie-de-tiempo}{%
\subsection{Desagregación de una serie de
tiempo}\label{desagregaciuxf3n-de-una-serie-de-tiempo}}

Tenemos una muestra de \(T\) observaciones de la variable aleatoria
\(Y_t\): \begin{equation}
    \{y_1, y_2,\dots, y_T \}
\end{equation}

\(Y_t\) tiene dos componentes: crecimiento (tendencia) \(s_t\) y ciclo
\(c_t\). \begin{equation}
    {y_t}{original} = {s_t}{tendencia} + {c_t}{ciclo}
\end{equation}

Asumimos que la tendencia es una curva \emph{suave}, aunque no
necesariamente una línea recta.

\texttt{\{figure\}\ figures/hp-filter-intuition.png}

\hypertarget{objetivos-en-conflicto}{%
\subsection{Objetivos en conflicto}\label{objetivos-en-conflicto}}

\begin{itemize}
\tightlist
\item
  Partiendo de \(y_t\), \cite{Hodrick-Prescott:1997} ``extraen'' la
  tendencia \(s_t\) \begin{equation*}
  \{s_1, s_2, \dots, s_T\},
  \end{equation*}
\end{itemize}

tratando de balancear dos objetivos mutuamente excluyentes:

\begin{enumerate}
\def\labelenumi{\arabic{enumi}.}
\tightlist
\item
  el ajuste a los datos originales, es decir, \(y_t-s_t\) debe ser
  pequeño.
\item
  la tendencia resultante debe ser suaver, por lo que los cambios de
  pendiente \((s_{t+1}-s_t)-(s_t-s_{t-1})\) también deben ser pequeños.
\end{enumerate}

La importancia relativa de estos dos factores es ponderada con el
parámetro \(\lambda\).

\hypertarget{el-filtro-de-hodrick-y-prescott-1}{%
\subsection{El filtro de Hodrick y
Prescott}\label{el-filtro-de-hodrick-y-prescott-1}}

Formalmente, la tendencia la definen por: \begin{align*}
s_i^{HP} &=   \argmin_{s_1,\dots,s_T}\left\{\sum_{t=1}^{T}\left(y_t-s_t\right)^2 +  \lambda \sum_{t=2}^{T-1}\left[\left(s_{t+1}-s_t\right) - \left(s_t-s_{t-1}\right)  \right]^2    \right\}\\ \\
         &= \argmin_{s_1,\dots,s_T}\left\{\sum_{t=1}^{T}\left(y_t-s_t\right)^2 +  \lambda \sum_{t=2}^{T-1}\left(s_{t+1}-2s_t + s_{t-1}\right)^2 \right\}
\end{align*}

\hypertarget{un-truco-de-uxe1lgebra-lineal}{%
\subsection{Un truco de álgebra
lineal}\label{un-truco-de-uxe1lgebra-lineal}}

Definimos las matrices \begin{equation*}
Y = \begin{bmatrix}
y_1 \\
y_2 \\
\vdots \\
y_T
\end{bmatrix}
\qquad
S = \begin{bmatrix}
s_1 \\
s_2 \\
\vdots \\
s_T
\end{bmatrix}
\end{equation*}

\begin{equation*}
A_{T-2\times T} =
\begin{bmatrix}
1 & -2 & 1 & 0 & \dots & 0 & 0 & 0 & 0 \\
0 & 1 & -2 & 1 & \dots & 0 & 0 & 0 & 0 \\
&  &  &  & \ddots &  &  &  &  \\
0 & 0 & 0 & 0 & \dots & 0 & 1 & -2 & 1
\end{bmatrix}
\end{equation*}

Reescribimos el problema de optimización \begin{align*}
    s_i^{HP} &=  \argmin_{s_1,\dots,s_T}\left\{\sum_{t=1}^{T}\left(y_t-s_t\right)^2 +  \lambda \sum_{t=2}^{T-1}\left(s_{t+1}-2s_t + s_{t-1}\right)^2 \right\}\\  
           &= \argmin_{S}\left\{(Y-S)'(Y-S) +  \lambda (AS)'(AS) \right\} \\
           &= \argmin_{S}\left\{Y'Y - 2Y'S +  S'(I + \lambda A'A)S \right\}
\end{align*}

\hypertarget{resolviendo-el-problema}{%
\subsection{Resolviendo el problema}\label{resolviendo-el-problema}}

Las condiciones de primer orden son \begin{align*}
S^{HP} &= \argmin_{S}\left\{Y'Y - 2Y'S +  S'(I + \lambda A'A)S \right\} \\
\Rightarrow & - 2Y + 2\left(I + \lambda A'A\right)S = 0
\end{align*}

Por lo que el filtro HP es \begin{align*}
S^{HP}                 &= \left(I + \lambda A'A\right)^{-1}Y \tag{tendencia} \\
C^{HP} \equiv Y-S^{HP} &= \left[I - \left(I + \lambda A'A\right)^{-1}\right]Y \tag{ciclo}
\end{align*}

\{\{ empieza\_ejemplo \}\} El filtro HP \{\{ fin\_titulo\_ejemplo \}\}
Asumimos que tenemos \(T=5\) datos
\(Y=\left[y_1,y_2,y_3,y_4,y_5\right]'\) y que \(\lambda=4\). Los datos
de tendencia \(S=\left[s_1,s_2,s_3,s_4,s_5\right]'\) están dados por:

\begin{align*}
 s_1 &= & &0.67y_1 &+ & 0.36y_2 &+ & 0.13y_3 &- & 0.02y_4 &- & 0.14y_5 \\
 s_2 &= & &0.36y_1 &+ & 0.34y_2 &+ & 0.23y_3 &+ & 0.10y_4 &- & 0.02y_5 \\
 s_3 &= & &0.13y_1 &+ & 0.23y_2 &+ & 0.29y_3 &+ & 0.23y_4 &+ & 0.13y_5 \\
 s_4 &= &-&0.02y_1 &+ & 0.10y_2 &+ & 0.23y_3 &+ & 0.34y_4 &+ & 0.36y_5 \\
 s_5 &= &-&0.14y_1 &- & 0.02y_2 &+ & 0.13y_3 &+ & 0.36y_4 &+ & 0.67y_5
\end{align*}

Observe que cada dato de tendencia \(s_t\) es simplemente un promedio
ponderado de todos los datos en \(Y\). Además. algunas de las
ponderaciones son negativas!

Por otra parte, los datos del ciclo
\(C=\left[c_1,c_2,c_3,c_4,c_5\right]'\) están dados por:

\begin{align*}
c_1 &= & &0.33y_1 &- & 0.36y_2 &- & 0.13y_3 &+ & 0.02y_4 &+ & 0.14y_5 \\
c_2 &= &-&0.36y_1 &+ & 0.66y_2 &- & 0.23y_3 &- & 0.10y_4 &+ & 0.02y_5 \\
c_3 &= &-&0.13y_1 &- & 0.23y_2 &+ & 0.71y_3 &- & 0.23y_4 &- & 0.13y_5 \\
c_4 &= & &0.02y_1 &- & 0.10y_2 &- & 0.23y_3 &+ & 0.66y_4 &- & 0.36y_5 \\
c_5 &= & &0.14y_1 &+ & 0.02y_2 &- & 0.13y_3 &- & 0.36y_4 &+ & 0.33y_5 \\
\end{align*}

De nuevo, observe que cada dato del ciclo \(c_t\) es un promedio
ponderado de todos los puntos en \(Y\), pero donde las ponderaciones
suman cero.

\{\{ termina\_ejemplo \}\}

\hypertarget{escogiendo-lambda}{%
\subsection{\texorpdfstring{Escogiendo
\(\lambda\)}{Escogiendo \textbackslash lambda}}\label{escogiendo-lambda}}

El resultado del filtro es muy sensible a la escogencia de \(\lambda\).
Como regla habitual, \(\lambda\) se escoge según la frecuencia de los
datos

\begin{itemize}
\tightlist
\item
  Anuales \(\Rightarrow 100\)
\item
  Trimestrales \(\Rightarrow 1600\)
\item
  Mensuales \(\Rightarrow 14400\)
\end{itemize}

\{\{ empieza\_ejemplo \}\} Filtered series when \(\lambda=1600\) \{\{
fin\_titulo\_ejemplo \}\}

    \begin{tcolorbox}[breakable, size=fbox, boxrule=1pt, pad at break*=1mm,colback=cellbackground, colframe=cellborder]
\prompt{In}{incolor}{2}{\boxspacing}
\begin{Verbatim}[commandchars=\\\{\}]
\PY{k}{def}\PY{+w}{ }\PY{n+nf}{HP\PYZus{}filter}\PY{p}{(}\PY{n}{y}\PY{p}{:} \PY{n}{pd}\PY{o}{.}\PY{n}{Series}\PY{p}{,} \PY{n}{𝜆}\PY{p}{:} \PY{n+nb}{float}\PY{p}{)}\PY{p}{:}
\PY{+w}{    }\PY{l+s+sd}{\PYZdq{}\PYZdq{}\PYZdq{}}
\PY{l+s+sd}{    Obtiene la tendencia y ciclo según filtro de Hodrick\PYZhy{}Prescott}

\PY{l+s+sd}{    A la serie se le calcula el logaritmo}

\PY{l+s+sd}{    :param x: datos originales}
\PY{l+s+sd}{    :param 𝜆: parámetro de suavizamiento de la tendencia}
\PY{l+s+sd}{    :return: un pd.DataFrame con la serie original, la tendencia, y el ciclo}
\PY{l+s+sd}{    \PYZdq{}\PYZdq{}\PYZdq{}}
    \PY{n}{y}\PY{o}{.}\PY{n}{dropna}\PY{p}{(}\PY{n}{inplace}\PY{o}{=}\PY{k+kc}{True}\PY{p}{)}
    \PY{n}{T} \PY{o}{=} \PY{n}{y}\PY{o}{.}\PY{n}{shape}\PY{p}{[}\PY{l+m+mi}{0}\PY{p}{]}
    \PY{n}{A} \PY{o}{=} \PY{n}{np}\PY{o}{.}\PY{n}{zeros}\PY{p}{(}\PY{p}{(}\PY{n}{T}\PY{o}{\PYZhy{}}\PY{l+m+mi}{2}\PY{p}{,} \PY{n}{T}\PY{p}{)}\PY{p}{)}
    \PY{k}{for} \PY{n}{i} \PY{o+ow}{in} \PY{n+nb}{range}\PY{p}{(}\PY{n}{T}\PY{o}{\PYZhy{}}\PY{l+m+mi}{2}\PY{p}{)}\PY{p}{:}
        \PY{n}{A}\PY{p}{[}\PY{n}{i}\PY{p}{,} \PY{n}{i}\PY{p}{:}\PY{n}{i}\PY{o}{+}\PY{l+m+mi}{3}\PY{p}{]} \PY{o}{=} \PY{l+m+mi}{1}\PY{p}{,} \PY{o}{\PYZhy{}}\PY{l+m+mi}{2}\PY{p}{,} \PY{l+m+mi}{1}
    \PY{n}{B} \PY{o}{=} \PY{n}{np}\PY{o}{.}\PY{n}{identity}\PY{p}{(}\PY{n}{T}\PY{p}{)} \PY{o}{+} \PY{n}{𝜆} \PY{o}{*} \PY{p}{(}\PY{n}{A}\PY{o}{.}\PY{n}{T} \PY{o}{@} \PY{n}{A}\PY{p}{)}
    \PY{n}{tendencia} \PY{o}{=} \PY{n}{np}\PY{o}{.}\PY{n}{exp}\PY{p}{(}\PY{n}{np}\PY{o}{.}\PY{n}{linalg}\PY{o}{.}\PY{n}{solve}\PY{p}{(}\PY{n}{B}\PY{p}{,} \PY{n}{np}\PY{o}{.}\PY{n}{log}\PY{p}{(}\PY{n}{y}\PY{p}{)}\PY{p}{)}\PY{p}{)}
    \PY{n}{ciclo} \PY{o}{=} \PY{l+m+mi}{100} \PY{o}{*} \PY{p}{(}\PY{n}{y}\PY{o}{/}\PY{n}{tendencia} \PY{o}{\PYZhy{}} \PY{l+m+mi}{1}\PY{p}{)}  \PY{c+c1}{\PYZsh{} como desviación porcentual respecto de la tendencia}

    \PY{k}{return} \PY{n}{pd}\PY{o}{.}\PY{n}{DataFrame}\PY{p}{(}
         \PY{p}{\PYZob{}}\PY{l+s+s1}{\PYZsq{}}\PY{l+s+s1}{serie original}\PY{l+s+s1}{\PYZsq{}}\PY{p}{:} \PY{n}{y}\PY{p}{,}
          \PY{l+s+s1}{\PYZsq{}}\PY{l+s+s1}{tendencia}\PY{l+s+s1}{\PYZsq{}}\PY{p}{:} \PY{n}{tendencia}\PY{p}{,}
          \PY{l+s+s1}{\PYZsq{}}\PY{l+s+s1}{ciclo}\PY{l+s+s1}{\PYZsq{}}\PY{p}{:} \PY{n}{ciclo}
         \PY{p}{\PYZcb{}}\PY{p}{,}
         \PY{n}{index} \PY{o}{=} \PY{n}{y}\PY{o}{.}\PY{n}{index}\PY{p}{)}



\PY{n}{gdp} \PY{o}{=} \PY{n}{pdr}\PY{o}{.}\PY{n}{get\PYZus{}data\PYZus{}fred}\PY{p}{(}\PY{l+s+s1}{\PYZsq{}}\PY{l+s+s1}{GDPC1}\PY{l+s+s1}{\PYZsq{}}\PY{p}{,} \PY{n}{start}\PY{o}{=}\PY{l+m+mi}{1947}\PY{p}{)}   

\PY{n}{gdp} \PY{o}{=} \PY{n}{HP\PYZus{}filter}\PY{p}{(}\PY{n}{gdp}\PY{p}{[}\PY{l+s+s1}{\PYZsq{}}\PY{l+s+s1}{GDPC1}\PY{l+s+s1}{\PYZsq{}}\PY{p}{]}\PY{p}{,} \PY{n}{𝜆}\PY{o}{=}\PY{l+m+mi}{1600}\PY{p}{)}

\PY{n}{gdp}\PY{p}{[}\PY{p}{[}\PY{l+s+s1}{\PYZsq{}}\PY{l+s+s1}{serie original}\PY{l+s+s1}{\PYZsq{}}\PY{p}{,} \PY{l+s+s1}{\PYZsq{}}\PY{l+s+s1}{tendencia}\PY{l+s+s1}{\PYZsq{}}\PY{p}{]}\PY{p}{]}\PY{o}{.}\PY{n}{plot}\PY{p}{(}\PY{n}{title}\PY{o}{=}\PY{l+s+s2}{\PYZdq{}}\PY{l+s+s2}{Serie desestacionalizada y su tendencia}\PY{l+s+s2}{\PYZdq{}}\PY{p}{)}

\PY{n}{gdp}\PY{p}{[}\PY{p}{[}\PY{l+s+s1}{\PYZsq{}}\PY{l+s+s1}{ciclo}\PY{l+s+s1}{\PYZsq{}}\PY{p}{]}\PY{p}{]}\PY{o}{.}\PY{n}{plot}\PY{p}{(}\PY{n}{title}\PY{o}{=}\PY{l+s+s2}{\PYZdq{}}\PY{l+s+s2}{Ciclo del PIB, como }\PY{l+s+si}{\PYZpc{} d}\PY{l+s+s2}{e desviación}\PY{l+s+s2}{\PYZdq{}}\PY{p}{)} 
\end{Verbatim}
\end{tcolorbox}

    
    
    
    
    Fuente de datos: \url{https://fred.stlouisfed.org/series/GDPC1/}

\{\{ termina\_ejemplo \}\}

\hypertarget{el-filtro-hp-tiene-malas-propiedades-estaduxedsticas}{%
\subsection{El filtro HP tiene malas propiedades
estadísticas}\label{el-filtro-hp-tiene-malas-propiedades-estaduxedsticas}}

\cite{Hamilton:2017}: Why You Should Never Use the HP Filter?

\begin{enumerate}
\def\labelenumi{\arabic{enumi}.}
\tightlist
\item
  El filtro Hodrick-Prescott introduce relaciones dinámicas espurias que
  no tienen sustento en el proceso generador de datos subyacente.
\item
  Los valores filtrados al final de la muestra son muy distintos de los
  del medio, y también están caracterizados por una dinámica espuria.
\item
  Una formalización estadística del problema típicamente produce valores
  de \(\lambda\) que distan mucho de los usados comúnmente en la
  práctica.
\item
  Para Hamilton, hay una alternativa mejor: una regresión AR(4) alcanza
  todos los objetivos buscados por usuarios del filtro HP pero con
  ninguno de sus desventajas.
\end{enumerate}

\{\{ empieza\_ejemplo \}\} Filtrando el PIB de Costa Rica con HP \{\{
fin\_titulo\_ejemplo \}\} Los datos filtrados son muy sensibles a nueva
información.

```\{figure\} figures/pib-hp-tails.png

Ciclo del PIB de Costa Rica, conforme se van agregando nuevas
observaciones ``` \{\{ termina\_ejemplo \}\}

\hypertarget{hp-puede-inducir-conclusiones-equivocadas-acerca-del-comovimiento-de-series}{%
\subsection{HP puede inducir conclusiones equivocadas acerca del
comovimiento de
series}\label{hp-puede-inducir-conclusiones-equivocadas-acerca-del-comovimiento-de-series}}

\textcite{CogleyNason:1995} analizaron las propiedades espectrales del
filtro HP

Cuando se mide el componente cíclico de una serie de tiempo, ¿es buena
idea usar el filtro HP?

Depende de la serie original

\begin{itemize}
\tightlist
\item
  \textbf{Sí}, si es estacionaria alrededor de tendencia
\item
  \textbf{No}, si es estacionaria en diferencia
\end{itemize}

Este resultado tiene implicaciones importantes para modelos DSGE: Cuando
se aplica el filtro HP a una serie integrada, el filtro introduce
periodicidad y comovimiento en las frecuencias del ciclo económico,
\textbf{aún si no estaban presentes en los datos originales}.


    % Add a bibliography block to the postdoc
    
    
    
\end{document}
