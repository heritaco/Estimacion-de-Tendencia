\documentclass[11pt,a4paper]{article}

% ---------------------------------------------------------
% Paquetes básicos
% ---------------------------------------------------------
\usepackage[spanish]{babel}
\usepackage[utf8]{inputenc}
\usepackage[T1]{fontenc}

\usepackage{amsmath, amssymb, amsfonts, amsthm}
\usepackage{mathtools}
\usepackage{bm}
\usepackage{geometry}
\geometry{margin=2.5cm}

% ---------------------------------------------------------
% Entornos teoréticos (si ya los tienes, puedes omitir esto)
% ---------------------------------------------------------
\newtheorem{definition}{Definición}[section]
\newtheorem{lemma}{Lema}[section]
\newtheorem{proposition}{Proposición}[section]
\newtheorem{remark}{Comentario}[section]

% ---------------------------------------------------------
% Comandos útiles
% ---------------------------------------------------------
\newcommand{\R}{\mathbb{R}}
\newcommand{\E}{\mathbb{E}}
\newcommand{\Var}{\mathbb{V}\mathrm{ar}}
\newcommand{\1}{\mathbf{1}}

\begin{document}

\title{Suavizamiento Exponencial Triple de Holt--Winters\\
con Selección de Parámetros por Validación sin Fuga de Información}
\author{(Borrador técnico)}
\date{}
\maketitle

\section{Planteamiento del problema y notación}

Sea
\[
\{P_t\}_{t=1}^N
\]
una serie de precios diarios ajustados de un activo financiero (acción, índice, tipo de cambio, cripto, etc.). Definimos la serie de \emph{log--precios}
\[
Z_t = \log P_t,\qquad t = 1,\dots,N.
\]

\subsection*{Por qué usar log--precios}

Trabajar con $Z_t = \log P_t$ en lugar de $P_t$ tiene varias ventajas:

\begin{itemize}
  \item Los rendimientos simples se aproximan por diferencias de log:
  \[
    r_t = \frac{P_t - P_{t-1}}{P_{t-1}}
    \approx \log P_t - \log P_{t-1} = Z_t - Z_{t-1}.
  \]
  \item Transformaciones multiplicativas en precios se vuelven aditivas en log--precios, lo cual es coherente con el modelo aditivo de Holt--Winters.
  \item Frecuentemente los errores en $Z_t$ son más cercanos a normales y de varianza aproximadamente constante que los errores en $P_t$.
\end{itemize}

\subsection*{Partición en entrenamiento, validación y prueba}

Sea $N$ el número total de observaciones. Fijamos proporciones
\[
0 < f_{\text{train}}, f_{\text{val}} < 1,\qquad 
f_{\text{train}} + f_{\text{val}} < 1,
\]
y definimos los tamaños enteros
\[
N_{\text{train}} = \lfloor f_{\text{train}} N \rfloor,\qquad
N_{\text{val}}   = \lfloor f_{\text{val}} N \rfloor,\qquad
N_{\text{test}}  = N - N_{\text{train}} - N_{\text{val}}.
\]

Partimos así la serie
\[
Z_1,\dots,Z_N
\]
en tres segmentos consecutivos:
\begin{align*}
  \text{Entrenamiento: } & Z_1,\dots,Z_{N_{\text{train}}},\\
  \text{Validación: }    & Z_{N_{\text{train}}+1},\dots,Z_{N_{\text{train}}+N_{\text{val}}},\\
  \text{Prueba: }        & Z_{N_{\text{train}}+N_{\text{val}}+1},\dots,Z_N.
\end{align*}

\begin{definition}[Sin fuga de información]
Diremos que un procedimiento de ajuste y selección de hiperparámetros no tiene \emph{fuga de información} si:
\begin{enumerate}
  \item Los parámetros del modelo (en este caso los estados de Holt--Winters: nivel, tendencia y estacionalidad) se actualizan exclusivamente con el segmento de entrenamiento.
  \item El segmento de validación se usa únicamente para evaluar pronósticos verdaderamente fuera de muestra.
  \item El segmento de prueba se mantiene completamente fuera de la selección de hiperparámetros y sólo se utiliza para una evaluación final.
\end{enumerate}
\end{definition}

Nuestro objetivo es ajustar un modelo de suavizamiento exponencial triple (Holt--Winters aditivo) sobre $\{Z_t\}$, elegir:
\[
s = \alpha \in (0,1),\quad m \in \{m_1,\dots,m_K\}
\]
mediante el error cuadrático medio (MSE) en validación, y producir pronósticos para la parte de validación y prueba, todo sin fuga de información.

\section{Modelo Holt--Winters aditivo (triple suavizamiento exponencial)}

\subsection{Descomposición aditiva}

El modelo Holt--Winters aditivo postula que el log--precio $Z_t$ puede escribirse como:
\[
Z_t = \ell_t + b_t + s_{p_t} + \varepsilon_t,\qquad t=1,\dots,N,
\]
donde:

\begin{itemize}
  \item $\ell_t$ es el \textbf{nivel} (tendencia lenta) en el tiempo $t$.
  \item $b_t$ es el \textbf{componente de crecimiento} o \emph{trend} local (pendiente).
  \item $s_{p_t}$ es el \textbf{componente estacional aditivo} en la posición $p_t$ dentro del ciclo estacional.
  \item $\varepsilon_t$ es un término de error (ruido) con media cero.
  \item $p_t$ es el índice estacional:
  \[
    p_t = t \bmod m \in \{0,1,\dots,m-1\},
  \]
  donde $m$ es la longitud del periodo estacional (por ejemplo, $m=5$ para “patrón semanal” en días hábiles).
\end{itemize}

Interpretación: en cada instante $t$, la serie es la suma de un nivel global, una tendencia lineal local y una corrección estacional que depende únicamente de la posición dentro del ciclo de longitud $m$.

\subsection{Parámetros de suavizamiento}

En una versión simplificada usamos un sólo hiperparámetro de suavizamiento $\alpha \in (0,1)$ y lo reutilizamos para:
\begin{itemize}
  \item Actualización del nivel,
  \item Actualización del componente de tendencia,
  \item Actualización del componente estacional.
\end{itemize}

Esto implica fijar
\[
\beta = \alpha,\qquad \gamma = \alpha,
\]
aunque en versiones más generales se podría tratar a $\beta$ y $\gamma$ como parámetros independientes.

\section{Inicialización de nivel, tendencia y estacionalidad}

Sea $m$ un candidato a longitud estacional. Suponemos que el conjunto de entrenamiento tiene al menos $2m$ observaciones:
\[
N_{\text{train}} \ge 2m,
\]
de modo que existe al menos dos ciclos completos dentro del entrenamiento.

\begin{definition}[Inicialización con promedios estacionales]
Sea
\[
Z_1,\dots,Z_{N_{\text{train}}}
\]
la serie de entrenamiento y $m$ la longitud estacional. Definimos:
\begin{itemize}
  \item $n_{\text{seas}} = \left\lfloor N_{\text{train}} / m \right\rfloor$ número de estaciones completas;
  \item elegimos un número $n_{\text{use}}$ de estaciones a usar para inicialización, típicamente
  \[
    n_{\text{use}} = \max\{2, \min(n_{\text{seas}}, 8)\};
  \]
  \item Para cada estación $k = 0,\dots,n_{\text{use}}-1$ tomamos el bloque
  \[
    \{ Z_{km+1},\dots,Z_{km+m} \}
  \]
  y definimos el promedio de la $k$--ésima estación
  \[
    \bar Z^{(k)} = \frac{1}{m} \sum_{j=1}^{m} Z_{km + j}.
  \]
\end{itemize}
\end{definition}

A partir de esto definimos:

\begin{itemize}
  \item \textbf{Nivel inicial}:
  \[
    \ell_0 = \frac{1}{n_{\text{use}}} \sum_{k=0}^{n_{\text{use}}-1} \bar Z^{(k)}.
  \]
  Es el promedio de las estaciones iniciales.

  \item \textbf{Tendencia inicial} (crecimiento promedio por unidad de tiempo):
  \[
    b_0 = \frac{\bar Z^{(1)} - \bar Z^{(0)}}{m},
  \]
  es decir, la diferencia entre las medias de las primeras dos estaciones, dividida por $m$ (número de pasos entre ellas). Si sólo hay una estación útil, podemos tomar $b_0 = 0$ como aproximación.

  \item \textbf{Índices estacionales iniciales} aditivos:
  para cada posición $j = 0,\dots,m-1$ dentro del ciclo, tomamos
  \[
    s_j = \frac{1}{n_{\text{use}}} \sum_{k=0}^{n_{\text{use}}-1} \left( Z_{km + (j+1)} - \bar Z^{(k)} \right).
  \]
  Es decir, la desviación promedio de la observación en la posición $j$ con respecto a la media de su estación.
\end{itemize}

Así obtenemos el vector inicial estacional
\[
\bm{s}^{(0)} = (s_0,\dots,s_{m-1}).
\]

\section{Actualización recursiva en el conjunto de entrenamiento}

Trabajamos ahora únicamente con las observaciones de entrenamiento
\[
Z_1,\dots,Z_{N_{\text{train}}},
\]
y mantenemos la \emph{restricción clave}: las observaciones de validación y prueba no se usan para actualizar los estados.

\subsection{Índice estacional}

Para cada tiempo $t\ge1$ definimos
\[
p_t = (t-1) \bmod m \in \{0,\dots,m-1\},
\]
de forma que la observación $Z_t$ corresponde a la posición $p_t$ dentro del ciclo estacional.

\subsection{Ecuaciones de Holt--Winters aditivo}

Con parámetros $\alpha \in (0,1)$, $\beta = \alpha$, $\gamma = \alpha$, el algoritmo de actualización es:

\begin{align}
  \text{Nivel:}\quad
  \ell_t
    &= \alpha\,(Z_t - s_{p_t}) + (1-\alpha)(\ell_{t-1} + b_{t-1}),
    \label{eq:level}\\[0.5em]
  \text{Tendencia:}\quad
  b_t
    &= \beta\,(\ell_t - \ell_{t-1}) + (1-\beta)\,b_{t-1},
    \label{eq:trend}\\[0.5em]
  \text{Estacionalidad:}\quad
  s_{p_t}
    &= \gamma\,(Z_t - \ell_t) + (1-\gamma)\,s_{p_t}.
    \label{eq:season}
\end{align}

\noindent Interpretación paso a paso:

\begin{itemize}
  \item En \eqref{eq:level}, la nueva estimación del nivel $\ell_t$ es una combinación convexa de:
    \begin{itemize}
      \item la observación \emph{deseasonalizada} $Z_t - s_{p_t}$;
      \item el nivel previo proyectado un paso adelante $\ell_{t-1}+b_{t-1}$.
    \end{itemize}

  \item En \eqref{eq:trend}, la nueva tendencia $b_t$ se actualiza como combinación convexa de:
    \begin{itemize}
      \item el \emph{incremento} observado del nivel $\ell_t - \ell_{t-1}$;
      \item la tendencia previa $b_{t-1}$.
    \end{itemize}

  \item En \eqref{eq:season}, el índice estacional en la posición $p_t$ se actualiza con:
    \begin{itemize}
      \item la desviación $Z_t - \ell_t$ entre la observación y el nivel,
      \item el valor estacional previo en esa misma posición.
    \end{itemize}
\end{itemize}

\subsection{Valores ajustados en entrenamiento}

Definimos el \emph{pronóstico un paso adelante} (ajuste in--sample) en el tiempo $t$ como
\[
\hat Z_t = \ell_{t-1} + b_{t-1} + s_{p_t},\qquad t\ge2.
\]
Para $t=1$ no hay historia anterior, así que se puede tomar
\[
\hat Z_1 = Z_1
\]
por conveniencia. De este modo se obtiene una serie ajustada
\[
\hat Z_1,\dots,\hat Z_{N_{\text{train}}}
\]
que suaviza la serie original de entrenamiento.

\section{Pronóstico fuera de muestra: validación y prueba}

Sea $T_{\text{train}} = N_{\text{train}}$ el último índice de entrenamiento. Después de procesar $Z_1,\dots,Z_{N_{\text{train}}}$ mediante \eqref{eq:level}--\eqref{eq:season}, tenemos:

\[
\ell_{T_{\text{train}}},\quad b_{T_{\text{train}}},\quad s_0,\dots,s_{m-1},\quad p_{T_{\text{train}}} = (T_{\text{train}}-1) \bmod m.
\]

Para un horizonte de pronóstico $h = 1,2,\dots$, el modelo Holt--Winters aditivo propone el pronóstico de $Z_{T_{\text{train}}+h}$ como
\begin{equation}
  \hat Z_{T_{\text{train}}+h \mid \text{train}}
  = \ell_{T_{\text{train}}} + h\,b_{T_{\text{train}}}
    + s_{p_{T_{\text{train}}+h}},
  \qquad p_{T_{\text{train}}+h} := (p_{T_{\text{train}}} + h) \bmod m.
  \label{eq:forecast-hw}
\end{equation}

\begin{definition}[Pronósticos de validación y prueba]
Para cada $k=1,\dots,N_{\text{val}}$ (validación) y $h = k$, se define
\[
\hat Z_{N_{\text{train}}+k \mid \text{train}}
 = \ell_{N_{\text{train}}} + k\,b_{N_{\text{train}}}
   + s_{p_{N_{\text{train}}+k}},
\]
y para el segmento de prueba, con $k' = 1,\dots,N_{\text{test}}$ (horizontes continuando desde el final de validación),
\[
\hat Z_{N_{\text{train}} + N_{\text{val}} + k' \mid \text{train}}
 = \ell_{N_{\text{train}}} + (N_{\text{val}}+k')\,b_{N_{\text{train}}}
   + s_{p_{N_{\text{train}} + N_{\text{val}} + k'}}.
\]
En todos los casos, ningún dato de validación ni de prueba modifica los estados del modelo.
\end{definition}

\section{Función objetivo en validación y búsqueda en rejilla}

\subsection{Función de pérdida en validación para $(\alpha,m)$}

Fijemos un candidato de suavización $s=\alpha\in(0,1)$ y un candidato de periodo estacional $m\in\mathbb{N}$. Procedemos así:

\begin{enumerate}
  \item Inicializar $(\ell_0,b_0,\bm{s}^{(0)})$ usando únicamente $Z_1,\dots,Z_{N_{\text{train}}}$ y $m$.
  \item Aplicar las actualizaciones \eqref{eq:level}--\eqref{eq:season} de $t=1$ a $t=N_{\text{train}}$.
  \item Obtener $(\ell_{N_{\text{train}}}, b_{N_{\text{train}}}, \bm{s})$.
  \item Generar los pronósticos de validación $\hat Z_{N_{\text{train}}+k\mid\text{train}}$ para $k=1,\dots,N_{\text{val}}$ usando \eqref{eq:forecast-hw}.
\end{enumerate}

Definimos los errores de validación para $(\alpha,m)$ como
\[
e_k(\alpha,m)
  := Z_{N_{\text{train}}+k}
   - \hat Z_{N_{\text{train}}+k\mid\text{train}} ,\qquad k=1,\dots,N_{\text{val}}.
\]

\begin{definition}[Error cuadrático medio de validación]
La función objetivo que queremos minimizar es el MSE de validación:
\[
J(\alpha,m)
  := \frac{1}{N_{\text{val}}} \sum_{k=1}^{N_{\text{val}}} \bigl(e_k(\alpha,m)\bigr)^2
  = \frac{1}{N_{\text{val}}} \sum_{k=1}^{N_{\text{val}}}
    \left( Z_{N_{\text{train}}+k}
      - \hat Z_{N_{\text{train}}+k\mid\text{train}} \right)^2.
\]
\end{definition}

Observación clave: \textbf{los pronósticos de validación se generan con el modelo entrenado únicamente en el segmento de entrenamiento}, por lo que $J(\alpha,m)$ es un verdadero error fuera de muestra.

\subsection{Rejilla de búsqueda en $\alpha$ y $m$}

Elegimos:

\begin{itemize}
  \item Una rejilla finita de suavizaciones:
  \[
    \alpha_1,\dots,\alpha_{K_\alpha} \in (0,1),
  \]
  típicamente un conjunto de puntos equiespaciados en $(0.01,0.99)$.

  \item Un conjunto discreto de periodos estacionales candidatos:
  \[
    \mathcal{M} = \{m_1,\dots,m_{K_m}\} \subset \mathbb{N},
  \]
  por ejemplo $\{5,10,20\}$ para tratar de capturar patrones semanales, quincenales o mensuales.
\end{itemize}

Definimos la rejilla total
\[
\mathcal{G} = \{(\alpha_k,m_i): k=1,\dots,K_\alpha,\ i=1,\dots,K_m\}.
\]

Para cada par $(\alpha_k,m_i)\in\mathcal{G}$:

\begin{enumerate}
  \item Verificamos que $N_{\text{train}} \ge 2m_i$; si no se cumple, no podemos inicializar correctamente, y marcamos $J(\alpha_k,m_i)=\text{NaN}$.
  \item En caso contrario, calculamos $J(\alpha_k,m_i)$ siguiendo el procedimiento anterior (actualizar sólo con entrenamiento y pronosticar validación).
\end{enumerate}

Finalmente, elegimos
\[
(\alpha^\ast,m^\ast)
  \in \arg\min_{(\alpha_k,m_i)\in\mathcal{G}} J(\alpha_k,m_i).
\]

Es decir, el par de parámetros que produce el menor MSE de validación dentro de la rejilla considerada.

\begin{remark}[Interpretación]
El periodo estacional $m^\ast$ se selecciona \emph{por evidencia empírica de validación}. Si la serie realmente tiene una estacionalidad fuerte en niveles (por ejemplo semanal), el $m$ correspondiente tenderá a generar un $J(\alpha,m)$ menor que otros valores. Si la serie es esencialmente no estacional en niveles, la componente estacional ajustada tenderá a ser pequeña y distintos $m$ producirán resultados similares.
\end{remark}

\section{Ajuste final y pronóstico sobre validación + prueba}

Una vez seleccionado $(\alpha^\ast,m^\ast)$:

\begin{enumerate}
  \item Reajustamos el modelo Holt--Winters aditivo en el conjunto de entrenamiento usando exactamente las mismas ecuaciones \eqref{eq:level}--\eqref{eq:season}, pero fijando $\alpha = \alpha^\ast$ y $m = m^\ast$.
  \item Obtenemos:
  \[
    \ell_{N_{\text{train}}}^{\ast},\quad
    b_{N_{\text{train}}}^{\ast},\quad
    \bm{s}^\ast,\quad
    p_{N_{\text{train}}}^\ast = (N_{\text{train}}-1) \bmod m^\ast.
  \]
  \item Definimos los pronósticos multi--paso, ahora tanto para validación como para prueba, como
  \[
  \hat Z_{N_{\text{train}} + h \mid \alpha^\ast, m^\ast}
   = \ell_{N_{\text{train}}}^{\ast}
     + h\, b_{N_{\text{train}}}^{\ast}
     + s^\ast_{(p_{N_{\text{train}}}^\ast + h)\bmod m^\ast},
   \qquad h = 1,\dots, N_{\text{val}}+N_{\text{test}}.
  \]
\end{enumerate}

Los errores en el conjunto de prueba se calculan como
\[
e^{\text{test}}_k =
  Z_{N_{\text{train}}+N_{\text{val}}+k}
  - \hat Z_{N_{\text{train}}+N_{\text{val}}+k\mid \alpha^\ast,m^\ast},
\qquad k=1,\dots,N_{\text{test}},
\]
y el MSE de prueba como
\[
\text{MSE}_{\text{test}}
 = \frac{1}{N_{\text{test}}} \sum_{k=1}^{N_{\text{test}}} \bigl(e^{\text{test}}_k\bigr)^2.
\]

\section{Resumen simbólico}

Resumiendo de forma compacta:

\begin{itemize}
  \item Dado $Z_t = \log P_t$, se propone el modelo aditivo
  \[
    Z_t = \ell_t + b_t + s_{p_t} + \varepsilon_t,\qquad p_t = (t-1)\bmod m.
  \]
  \item Se inicializa $(\ell_0,b_0,\bm{s})$ a partir de promedios de estaciones completas en el conjunto de entrenamiento.
  \item Sobre $\{Z_t\}_{t=1}^{N_{\text{train}}}$ se actualizan los estados mediante:
  \[
  \boxed{
  \begin{aligned}
    \ell_t
      &= \alpha\,(Z_t - s_{p_t}) 
         + (1-\alpha)(\ell_{t-1}+b_{t-1}),\\
    b_t
      &= \alpha\,(\ell_t-\ell_{t-1})
         + (1-\alpha)\,b_{t-1},\\
    s_{p_t}
      &= \alpha\,(Z_t - \ell_t)
         + (1-\alpha)\,s_{p_t}.
  \end{aligned}
  }
  \]
  \item El pronóstico multi--paso desde el final de entrenamiento es:
  \[
  \boxed{
    \hat Z_{N_{\text{train}}+h\mid\text{train}}
    = \ell_{N_{\text{train}}} + h\,b_{N_{\text{train}}}
      + s_{(p_{N_{\text{train}}}+h)\bmod m},
    \quad h\ge1.
  }
  \]
  \item La función objetivo de validación para un par $(\alpha,m)$ es
  \[
  \boxed{
    J(\alpha,m)
    = \frac{1}{N_{\text{val}}}
      \sum_{k=1}^{N_{\text{val}}}
      \bigl(
        Z_{N_{\text{train}}+k}
        - \hat Z_{N_{\text{train}}+k\mid\text{train}}
      \bigr)^2.
  }
  \]
  \item Se selecciona
  \[
  \boxed{
    (\alpha^\ast,m^\ast)
     \in \arg\min_{(\alpha,m)\in\mathcal{G}} J(\alpha,m).
  }
  \]
  \item Finalmente, se rehace el ajuste sobre entrenamiento con $(\alpha^\ast,m^\ast)$ y se usa el modelo resultante para pronosticar tanto la parte de validación como la de prueba, sin emplear nunca esos datos para reentrenar o actualizar estados (sin fuga de información).
\end{itemize}

\end{document}
